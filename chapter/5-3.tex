% Copyright 2018 Melvin Eloy Irizarry-Gelpí
\chapter{3-Exponential Functions}
%%%%%%%%%%%%%%%%%%%%%%%%%%%%%%%%%%%%%%%%%%%%%%%%%%%%%%%%%%%%%%%%%%%%%%%%%%%%%%%%
The three 3-exponential functions are the first and simplest generalization. These functions are more similar to the 1-exponential functions than to the 2-hyperbolic functions.
%%%%%%%%%%%%%%%%%%%%%%%%%%%%%%%%%%%%%%%%%%%%%%%%%%%%%%%%%%%%%%%%%%%%%%%%%%%%%%%%
\section{Definition}
%%%%%%%%%%%%%%%%%%%%%%%%%%%%%%%%%%%%%%%%%%%%%%%%%%%%%%%%%%%%%%%%%%%%%%%%%%%%%%%%
The 3-exponential functions are
\begin{align}
    B_{30}(z) & = \sum_{n = 0}^{\infty} \frac{1}{\Gamma(3n + 1)} z^{3n} \\
	B_{31}(z) & = \sum_{n = 0}^{\infty} \frac{1}{\Gamma(3n + 2)} z^{3n+1} \\
	B_{32}(z) & = \sum_{n = 0}^{\infty} \frac{1}{\Gamma(3n + 3)} z^{3n+2}
\end{align}
These functions can be written in terms of elementary functions as follows.

Consider the following identities:
\begin{align}
    \exp\left( z \right) &= B_{30}(z) + B_{31}(z) + B_{32}(z) \\
    \exp\left( \phi^{2} z \right) &= B_{30}(z) + \phi^{2} B_{31}(z) - \phi B_{32}(z) \\
    \exp\left( -\phi z \right) &= B_{30}(z) - \phi B_{31}(z) + \phi^{2} B_{32}(z)
\end{align}
Here $\phi$ is the complex number defined in (\ref{eq.2.phi}). This system of equations can be stated as a matrix equation:
\begin{equation}
    \begin{bmatrix}
        \exp\left( z \right) \\ \exp\left( \phi^{2} z \right) \\ \exp\left( -\phi z \right)
    \end{bmatrix} = \begin{bmatrix}
        1 & 1 & 1 \\ 1 & \phi^{2} & {-\phi} \\ 1 & {-\phi} & \phi^{2}
    \end{bmatrix}
    \begin{bmatrix}
        B_{30}(z) \\ B_{31}(z) \\ B_{32}(z)
    \end{bmatrix}
\end{equation}
You can recognize the matrix $E_{3}$ from (\ref{eq.2.E3}). Inverting this matrix gives
\begin{equation}
    \begin{bmatrix}
        B_{30}(z) \\ B_{31}(z) \\ B_{32}(z)
    \end{bmatrix} = \frac{1}{3} \begin{bmatrix}
        1 & 1 & 1 \\ 1 & {-\phi} & \phi^{2} \\ 1 & \phi^{2} & {-\phi}
    \end{bmatrix}
    \begin{bmatrix}
        \exp\left( z \right) \\ \exp\left( \phi^{2} z \right) \\ \exp\left( -\phi z \right)
    \end{bmatrix}
\end{equation}
You can recognize the matrix $F_{3}$ from (\ref{eq.2.F3}). Explicitly
\begin{align}
    B_{30}(z) &= \frac{1}{3} \left[ \exp(z) + \exp\left(\phi^{2} z\right) + \exp\left(-\phi z\right) \right] \\
    B_{31}(z) &= \frac{1}{3} \left[ \exp(z) - \phi \exp\left(\phi^{2} z\right) + \phi^{2}\exp\left(-\phi z\right) \right] \\
    B_{32}(z) &= \frac{1}{3} \left[ \exp(z) + \phi^{2} \exp\left(\phi^{2} z\right) - \phi \exp\left(-\phi z\right) \right]
\end{align}
Using the fact that
\begin{align}
    \exp\left( \phi^{2} z \right) &= \exp\left(-\frac{z}{2}\right) \left[ \cos\left( \frac{\sqrt{3}}{2}z \right) + i \sin\left( \frac{\sqrt{3}}{2}z \right) \right] \\
    \exp\left( -\phi z \right) &= \exp\left(-\frac{z}{2}\right) \left[ \cos\left( \frac{\sqrt{3}}{2}z \right) - i \sin\left( \frac{\sqrt{3}}{2}z \right) \right]
\end{align}
You find
\begin{align}
    B_{30}(z) &= \frac{1}{3} \exp(z) + \frac{2}{3} \exp\left( -\frac{z}{2} \right) \cos\left( \frac{\sqrt{3}}{2} z \right) \\
    B_{31}(z) &= \frac{1}{3} \exp(z) - \frac{1}{3} \exp\left( -\frac{z}{2} \right) \cos\left( \frac{\sqrt{3}}{2} z \right) + \frac{1}{\sqrt{3}} \exp\left( -\frac{z}{2} \right) \sin\left( \frac{\sqrt{3}}{2} z \right) \\
    B_{32}(z) &= \frac{1}{3} \exp(z) - \frac{1}{3} \exp\left( -\frac{z}{2} \right) \cos\left( \frac{\sqrt{3}}{2} z \right) - \frac{1}{\sqrt{3}} \exp\left( -\frac{z}{2} \right) \sin\left( \frac{\sqrt{3}}{2} z \right)
\end{align}
%%%%%%%%%%%%%%%%%%%%%%%%%%%%%%%%%%%%%%%%%%%%%%%%%%%%%%%%%%%%%%%%%%%%%%%%%%%%%%%%
\section{Properties}
%%%%%%%%%%%%%%%%%%%%%%%%%%%%%%%%%%%%%%%%%%%%%%%%%%%%%%%%%%%%%%%%%%%%%%%%%%%%%%%%
Here are some important properties.
%%%%%%%%%%%%%%%%%%%%%%%%%%%%%%%%%%%%%%%%%%%%%%%%%%%%%%%%%%%%%%%%%%%%%%%%%%%%%%%%
\subsection{Special Values}
%%%%%%%%%%%%%%%%%%%%%%%%%%%%%%%%%%%%%%%%%%%%%%%%%%%%%%%%%%%%%%%%%%%%%%%%%%%%%%%%
Note that
\begin{align}
    B_{30}(0) &= 1 \\
    B_{31}(0) &= 0 \\
    B_{32}(0) &= 0
\end{align}
%%%%%%%%%%%%%%%%%%%%%%%%%%%%%%%%%%%%%%%%%%%%%%%%%%%%%%%%%%%%%%%%%%%%%%%%%%%%%%%%
\subsection{Real and Imaginary Parts}
%%%%%%%%%%%%%%%%%%%%%%%%%%%%%%%%%%%%%%%%%%%%%%%%%%%%%%%%%%%%%%%%%%%%%%%%%%%%%%%%
...
%%%%%%%%%%%%%%%%%%%%%%%%%%%%%%%%%%%%%%%%%%%%%%%%%%%%%%%%%%%%%%%%%%%%%%%%%%%%%%%%
\subsection{Anti-Periodicity}
%%%%%%%%%%%%%%%%%%%%%%%%%%%%%%%%%%%%%%%%%%%%%%%%%%%%%%%%%%%%%%%%%%%%%%%%%%%%%%%%
...
%%%%%%%%%%%%%%%%%%%%%%%%%%%%%%%%%%%%%%%%%%%%%%%%%%%%%%%%%%%%%%%%%%%%%%%%%%%%%%%%
\subsection{Periodicity}
%%%%%%%%%%%%%%%%%%%%%%%%%%%%%%%%%%%%%%%%%%%%%%%%%%%%%%%%%%%%%%%%%%%%%%%%%%%%%%%%
...
%%%%%%%%%%%%%%%%%%%%%%%%%%%%%%%%%%%%%%%%%%%%%%%%%%%%%%%%%%%%%%%%%%%%%%%%%%%%%%%%
\section{Identities}
%%%%%%%%%%%%%%%%%%%%%%%%%%%%%%%%%%%%%%%%%%%%%%%%%%%%%%%%%%%%%%%%%%%%%%%%%%%%%%%%
Here are some important identities.
%%%%%%%%%%%%%%%%%%%%%%%%%%%%%%%%%%%%%%%%%%%%%%%%%%%%%%%%%%%%%%%%%%%%%%%%%%%%%%%%
\subsection{Addition Formulae}
%%%%%%%%%%%%%%%%%%%%%%%%%%%%%%%%%%%%%%%%%%%%%%%%%%%%%%%%%%%%%%%%%%%%%%%%%%%%%%%%
The addition formulae are
\begin{align}
    B_{30}(w + z) &= B_{30}(w) B_{30}(z) + B_{31}(w) B_{32}(z) + B_{32}(w) B_{31}(z) \\
    B_{31}(w + z) &= B_{30}(w) B_{31}(z) + B_{31}(w) B_{30}(z) + B_{32}(w) B_{32}(z) \\
    B_{32}(w + z) &= B_{30}(w) B_{32}(z) + B_{31}(w) B_{31}(z) + B_{32}(w) B_{30}(z)
\end{align}
These formulae can be found as follows. Let $V$ be a square matrix such that
\begin{equation}
    V^{3} = I
\end{equation}
with $I$ the identity matrix. Then
\begin{equation}
    \exp\left( z V \right) = I B_{30}(z) + V B_{31}(z) + V^{2} B_{32}(z)
\end{equation}
Using the identity
\begin{equation}
    \exp\left( w V \right) \exp\left( z V \right) = \exp\left( wV + zV \right)
\end{equation}
you can find the above formulae.
%%%%%%%%%%%%%%%%%%%%%%%%%%%%%%%%%%%%%%%%%%%%%%%%%%%%%%%%%%%%%%%%%%%%%%%%%%%%%%%%
\subsection{de Moivre Scaling Formulae}
%%%%%%%%%%%%%%%%%%%%%%%%%%%%%%%%%%%%%%%%%%%%%%%%%%%%%%%%%%%%%%%%%%%%%%%%%%%%%%%%
For the 3-exponential case, the de Moivre scale factor is
\begin{equation}
    r_{61} = \phi = \frac{1}{2} + \frac{\sqrt{3}}{2}i
\end{equation}
This leads to the following de Moivre scaling formulae:
\begin{align}
    B_{30}(\phi z) &= \sum_{n = 0}^{\infty} \frac{(-1)^{n}}{\Gamma\left(3n + 1\right)} z^{3n} = A_{30}(z) \\
    B_{31}(\phi z) &= \phi \sum_{n = 0}^{\infty} \frac{(-1)^{n}}{\Gamma\left(3n + 2\right)} z^{3n+1} = \phi A_{31}(z) \\
    B_{32}(\phi z) &= \phi^{2} \sum_{n = 0}^{\infty} \frac{(-1)^{n}}{\Gamma\left(3n + 3\right)} z^{3n+2} = \phi^{2} A_{30}(z)
\end{align}
%%%%%%%%%%%%%%%%%%%%%%%%%%%%%%%%%%%%%%%%%%%%%%%%%%%%%%%%%%%%%%%%%%%%%%%%%%%%%%%%
\subsection{Differentiation Formulae}
%%%%%%%%%%%%%%%%%%%%%%%%%%%%%%%%%%%%%%%%%%%%%%%%%%%%%%%%%%%%%%%%%%%%%%%%%%%%%%%%
The first-order differentiation formulae are
\begin{align}
    \partial B_{30} &= B_{32} & \partial B_{31} &= B_{30} & \partial B_{32} &= B_{31}
\end{align}
The second-order differentiation formulae are
\begin{align}
    \partial^{2} B_{30} &= B_{31} & \partial^{2} B_{31} &= B_{32} & \partial^{2} B_{32} &= B_{30}
\end{align}
The third-order differentiation formulae are
\begin{align}
    \partial^{3} B_{30} &= B_{30} & \partial^{3} B_{31} &= B_{31} & \partial^{3} B_{32} &= B_{32}
\end{align}
%%%%%%%%%%%%%%%%%%%%%%%%%%%%%%%%%%%%%%%%%%%%%%%%%%%%%%%%%%%%%%%%%%%%%%%%%%%%%%%%
\subsection{Wronskians}
%%%%%%%%%%%%%%%%%%%%%%%%%%%%%%%%%%%%%%%%%%%%%%%%%%%%%%%%%%%%%%%%%%%%%%%%%%%%%%%%
With three 3-exponential functions, you can have three $2 \times 2$ Wronskians and one $3 \times 3$ Wronskian.

The $2 \times 2$ Wronskians are
\begin{align}
    W_{301} &= \begin{vmatrix}
        B_{30} & B_{31} \\ \partial B_{30} & \partial B_{31}
    \end{vmatrix} = \begin{vmatrix}
        B_{30} & B_{31} \\ B_{32} & B_{30}
    \end{vmatrix} = \left(B_{30}\right)^{2} - B_{31} B_{32} \\
    W_{312} &= \begin{vmatrix}
        B_{31} & B_{32} \\ \partial B_{31} & \partial B_{32}
    \end{vmatrix} = \begin{vmatrix}
        B_{31} & B_{32} \\ B_{30} & B_{31}
    \end{vmatrix} = \left(B_{31}\right)^{2} - B_{30} B_{32} \\
    W_{320} &= \begin{vmatrix}
        B_{32} & B_{30} \\ \partial B_{32} & \partial B_{30}
    \end{vmatrix} = \begin{vmatrix}
        B_{32} & B_{30} \\ B_{31} & B_{32}
    \end{vmatrix} = \left(B_{32}\right)^{2} - B_{30} B_{31}
\end{align}
The $3 \times 3$ Wronskian is
\begin{align}
    W_{3012} &= \begin{vmatrix}
        B_{30} & B_{31} & B_{32} \\ \partial B_{30} & \partial B_{31} & \partial B_{32} \\ \partial^{2} B_{30} & \partial^{2} B_{31} & \partial^{2} B_{32}
    \end{vmatrix} = \begin{vmatrix}
        B_{30} & B_{31} & B_{32} \\ B_{32} & B_{30} & B_{31} \\ B_{31} & B_{32} & B_{30}
    \end{vmatrix} \\
    &= \left( B_{30} \right)^{3} + \left( B_{31} \right)^{3} + \left( B_{32} \right)^{3} - 3 B_{30} B_{31} B_{32}
\end{align}
Note that all Wronskian matrices are Toeplitz matrices.
%%%%%%%%%%%%%%%%%%%%%%%%%%%%%%%%%%%%%%%%%%%%%%%%%%%%%%%%%%%%%%%%%%%%%%%%%%%%%%%%
\subsection{Algebraic Formulae}
%%%%%%%%%%%%%%%%%%%%%%%%%%%%%%%%%%%%%%%%%%%%%%%%%%%%%%%%%%%%%%%%%%%%%%%%%%%%%%%%
Recall that
\begin{equation}
    1 + \phi^{2} - \phi = 0
\end{equation}