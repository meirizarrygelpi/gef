% Copyright 2018 Melvin Eloy Irizarry-Gelpí
\chapter{Definitions}
%%%%%%%%%%%%%%%%%%%%%%%%%%%%%%%%%%%%%%%%%%%%%%%%%%%%%%%%%%%%%%%%%%%%%%%%%%%%%%%%
Here are some definitions.
%%%%%%%%%%%%%%%%%%%%%%%%%%%%%%%%%%%%%%%%%%%%%%%%%%%%%%%%%%%%%%%%%%%%%%%%%%%%%%%%
\section{Euler Gamma Function}
%%%%%%%%%%%%%%%%%%%%%%%%%%%%%%%%%%%%%%%%%%%%%%%%%%%%%%%%%%%%%%%%%%%%%%%%%%%%%%%%
The \textbf{Euler Gamma function} satisfies the property
\begin{equation}
	\Gamma(z + 1) = z \Gamma(z)
\end{equation}
This expression can be rearranged into
\begin{equation}
	\frac{1}{\Gamma(z)} = \frac{z}{\Gamma(z + 1)}
\end{equation}
This identity will be used extensively later.
%%%%%%%%%%%%%%%%%%%%%%%%%%%%%%%%%%%%%%%%%%%%%%%%%%%%%%%%%%%%%%%%%%%%%%%%%%%%%%%%
\section{Elementary Functions}
%%%%%%%%%%%%%%%%%%%%%%%%%%%%%%%%%%%%%%%%%%%%%%%%%%%%%%%%%%%%%%%%%%%%%%%%%%%%%%%%
A very quick review of three important kinds of elementary functions.
%%%%%%%%%%%%%%%%%%%%%%%%%%%%%%%%%%%%%%%%%%%%%%%%%%%%%%%%%%%%%%%%%%%%%%%%%%%%%%%%
\subsection{Exponential Function}
%%%%%%%%%%%%%%%%%%%%%%%%%%%%%%%%%%%%%%%%%%%%%%%%%%%%%%%%%%%%%%%%%%%%%%%%%%%%%%%%
The \textbf{exponential function} can be defined as a power series:
\begin{equation}
	\exp(z) = \sum_{n = 0}^{\infty} \frac{1}{\Gamma(n + 1)} z^{n}
\end{equation}
Using the above property of the Euler Gamma function, you find the differentiation formula:
\begin{equation}
	\partial \exp(z) = \exp(z)
\end{equation}
This expression can be rearranged into
\begin{equation}
	\left( \partial - 1 \right) \exp(z) = 0
\end{equation}
This is a first-order differential equation; there is only one linearly-independent solution.

The \textbf{reciprocal of the exponential function} can be defined as an \textbf{alternating} power series:
\begin{equation}
	\exp({-z}) = \sum_{n = 0}^{\infty} \frac{(-1)^{n}}{\Gamma(n + 1)} z^{n}
\end{equation}
Using the above property of the Euler Gamma function, you find the differentiation formula:
\begin{equation}
	\partial \exp({-z}) = -\exp({-z})
\end{equation}
This expression can be rearranged into
\begin{equation}
	\left( \partial + 1 \right) \exp({-z}) = 0
\end{equation}
This is a first-order differential equation; there is only one linearly-independent solution.
%%%%%%%%%%%%%%%%%%%%%%%%%%%%%%%%%%%%%%%%%%%%%%%%%%%%%%%%%%%%%%%%%%%%%%%%%%%%%%%%
\subsection{Hyperbolic Functions}
%%%%%%%%%%%%%%%%%%%%%%%%%%%%%%%%%%%%%%%%%%%%%%%%%%%%%%%%%%%%%%%%%%%%%%%%%%%%%%%%
The two basic \textbf{hyperbolic functions} can be defined as power series:
\begin{align}
	\cosh(z) & = \sum_{n = 0}^{\infty} \frac{1}{\Gamma(2n + 1)} z^{2n} & \sinh(z) & = \sum_{n = 0}^{\infty} \frac{1}{\Gamma(2n + 2)} z^{2n + 1}
\end{align}
Taking one derivative gives the differentiation formulae:
\begin{align}
	\partial \sinh(z) & = \cosh(z) & \partial \cosh(z) & = \sinh(z)
\end{align}
Taking two derivatives gives
\begin{align}
	\partial^{2} \cosh(z) & = \cosh(z) & \partial^{2} \sinh(z) & = \sinh(z)
\end{align}
These expressions can be rearranged into
\begin{align}
	\left( \partial^{2} - 1 \right) \cosh(z) & = 0 & \left( \partial^{2} - 1 \right) \sinh(z) & = 0
\end{align}
These are second-order differential equations. Indeed, the most general solution to the differential equation
\begin{equation}
	\left( \partial^{2} - 1 \right) f(z) = 0
\end{equation}
is
\begin{equation}
	f(z) = b_{0} \cosh(z) + b_{1} \sinh(z)
\end{equation}
These two hyperbolic functions are examples of elementary functions.
%%%%%%%%%%%%%%%%%%%%%%%%%%%%%%%%%%%%%%%%%%%%%%%%%%%%%%%%%%%%%%%%%%%%%%%%%%%%%%%%
\subsection{Trigonometric Functions}
%%%%%%%%%%%%%%%%%%%%%%%%%%%%%%%%%%%%%%%%%%%%%%%%%%%%%%%%%%%%%%%%%%%%%%%%%%%%%%%%
The two basic \textbf{trigonometric functions} can be defined as alternating power series:
\begin{align}
	\cos(z) & = \sum_{n = 0}^{\infty} \frac{(-1)^{n}}{\Gamma(2n + 1)} z^{2n} & \sin(z) & = \sum_{n = 0}^{\infty} \frac{(-1)^{n}}{\Gamma(2n + 2)} z^{2n + 1}
\end{align}
Taking one derivative gives the differentiation formulae:
\begin{align}
	\partial \cos(z) & = -\sin(z) & \partial \sin(z) & = \cos(z)
\end{align}
Taking two derivatives gives
\begin{align}
	\partial^{2} \cos(z) & = -\cos(z) & \partial^{2} \sin(z) & = -\sin(z)
\end{align}
These expressions can be rearranged into
\begin{align}
	\left( \partial^{2} + 1 \right) \cos(z) & = 0 & \left( \partial^{2} + 1 \right) \sin(z) & = 0
\end{align}
These are second-order differential equations. Indeed, the most general solution to the differential equation
\begin{equation}
	\left( \partial^{2} + 1 \right) f(z) = 0
\end{equation}
is
\begin{equation}
	f(z) = a_{0} \cos(z) + a_{1} \sin(z)
\end{equation}
These two trigonometric functions are also examples of elementary functions.
%%%%%%%%%%%%%%%%%%%%%%%%%%%%%%%%%%%%%%%%%%%%%%%%%%%%%%%%%%%%%%%%%%%%%%%%%%%%%%%%
\section{Mittag-Leffler Function}
%%%%%%%%%%%%%%%%%%%%%%%%%%%%%%%%%%%%%%%%%%%%%%%%%%%%%%%%%%%%%%%%%%%%%%%%%%%%%%%%
The \textbf{Mittag-Leffler function} is defined as the power series
\begin{equation}
	\operatorname{ML}(\alpha, \beta; z) \equiv \sum_{n = 0}^{\infty} \frac{1}{\Gamma(\alpha n + \beta)} z^{n}
\end{equation}
All the elementary functions introduced above are special cases of the ML function:
\begin{align}
	\exp(z)  & = \operatorname{ML}(1, 1; z)                 \\
	\cosh(z) & = \operatorname{ML}\left(2, 1; z^{2}\right)  \\
	\sinh(z) & = \operatorname{ML}\left(2, 2; z^{2}\right)  \\
	\exp(-z) & = \operatorname{ML}(1, 1; -z)                \\
	\cos(z)  & = \operatorname{ML}\left(2, 1; -z^{2}\right) \\
	\sin(z)  & = \operatorname{ML}\left(2, 2; -z^{2}\right)
\end{align}
The ML function satisfies many relations. For example, multiplication by the argument variable gives:
\begin{equation}
	z \operatorname{ML}(\alpha, \beta; z) = \operatorname{ML}(\alpha, \beta - \alpha; z) - \frac{1}{\Gamma(\beta - \alpha)}
\end{equation}
This expression can be rearranged into
\begin{equation}
	\operatorname{ML}(\alpha, \beta - \alpha; z) = z \operatorname{ML}(\alpha, \beta; z) + \frac{1}{\Gamma(\beta - \alpha)}
\end{equation}
In a similar way, for $k \geq 1$ you get
\begin{equation}
	z^{k} \operatorname{ML}(\alpha, \beta; z) = \operatorname{ML}(\alpha, \beta - k\alpha; z) - \sum_{n = 0}^{k-1} \frac{1}{\Gamma(\alpha n + \beta - k \alpha)} z^{n}
\end{equation}
Division by the argument variable gives another set of relations:
\begin{equation}
	\frac{1}{z} \operatorname{ML}(\alpha, \beta; z) = \operatorname{ML}(\alpha, \beta + \alpha; z) + \frac{1}{z \Gamma(\beta)}
\end{equation}
This expression can be rearranged into
\begin{equation}
	\operatorname{ML}(\alpha, \beta; z) = z \operatorname{ML}(\alpha, \beta + \alpha; z) + \frac{1}{\Gamma(\beta)}
\end{equation}
In a similar way, for $k \geq 1$ you get
\begin{equation}
	\frac{1}{z^{k}} \operatorname{ML}(\alpha, \beta; z) = \operatorname{ML}(\alpha, \beta + k \alpha; z) + \frac{1}{z^{k}} \sum_{n = 0}^{k - 1} \frac{1}{\Gamma(\alpha n + \beta)} z^{n}
\end{equation}
Or equivalently
\begin{equation}
	\operatorname{ML}(\alpha, \beta; z) = z^{k} \operatorname{ML}(\alpha, \beta + k \alpha; z) + \sum_{n = 0}^{k - 1} \frac{1}{\Gamma(\alpha n + \beta)} z^{n}
\end{equation}
%%%%%%%%%%%%%%%%%%%%%%%%%%%%%%%%%%%%%%%%%%%%%%%%%%%%%%%%%%%%%%%%%%%%%%%%%%%%%%%%
\section{Generalized Mittag-Leffler Functions}
%%%%%%%%%%%%%%%%%%%%%%%%%%%%%%%%%%%%%%%%%%%%%%%%%%%%%%%%%%%%%%%%%%%%%%%%%%%%%%%%
The \textbf{generalized Mittag-Leffler functions} are defined as the power series:
\begin{align}
	\operatorname{GML}(a, b, c, d; z) & \equiv \sum_{n = 0}^{\infty} \frac{1}{\Gamma(a n + b + 1)} z^{cn+d}        \\
	\operatorname{AML}(a, b, c, d; z) & \equiv \sum_{n = 0}^{\infty} \frac{(-1)^{n}}{\Gamma(a n + b + 1)} z^{cn+d}
\end{align}
In terms of GML, the ML function can be written as
\begin{equation}
	\operatorname{ML}(\alpha, \beta; z) = \operatorname{GML}(\alpha, \beta - 1, 1, 0; z)
\end{equation}
Conversely, in terms of ML, the GML and AML functions can be written as
\begin{align}
	\operatorname{GML}(a, b, c, d; z) & = z^{d} \operatorname{ML}\left(a, b+1; z^{c}\right)  \\
	\operatorname{AML}(a, b, c, d; z) & = z^{d} \operatorname{ML}\left(a, b+1; -z^{c}\right)
\end{align}
In terms of GML and AML, the elementary functions above are given by
\begin{align}
	\exp(z)  & = \operatorname{GML}(1, 0, 1, 0; z) \\
	\cosh(z) & = \operatorname{GML}(2, 0, 1, 0; z) \\
	\sinh(z) & = \operatorname{GML}(2, 1, 2, 1; z) \\
	\exp(-z) & = \operatorname{AML}(1, 0, 1, 0; z) \\
	\cos(z)  & = \operatorname{AML}(2, 0, 1, 0; z) \\
	\sin(z)  & = \operatorname{AML}(2, 1, 2, 1; z)
\end{align}
For simplicity, define the following power series:
\begin{align}
	B_{pq}(z) & \equiv \operatorname{GML}(p, q, p, q; z) = \sum_{n = 0}^{\infty} \frac{1}{\Gamma(p n + q + 1)} z^{pn + q}        \\
	A_{pq}(z) & \equiv \operatorname{AML}(p, q, p, q; z) = \sum_{n = 0}^{\infty} \frac{(-1)^{n}}{\Gamma(p n + q + 1)} z^{pn + q}
\end{align}
These functions are solutions to $p$-order differential equations:
\begin{align}
	\left( \partial^{p} - 1 \right) B_{pq}(z) & = 0 & \left( \partial^{p} + 1 \right) A_{pq}(z) & = 0 & q & = 0, 1, ..., p -1
\end{align}
Indeed, if
\begin{align}
	f(z) & = \sum_{q = 0}^{p-1} b_{q} B_{pq}(z) \\
	g(z) & = \sum_{q = 0}^{p-1} a_{q} A_{pq}(z)
\end{align}
then
\begin{align}
	\left( \partial^{p} - 1 \right) f(z) & = 0 \\
	\left( \partial^{p} + 1 \right) g(z) & = 0
\end{align}
In terms of these functions, you have
\begin{align}
	\exp(z)  & = B_{10}(z) \\
	\cosh(z) & = B_{20}(z) \\
	\sinh(z) & = B_{21}(z) \\
	\exp(-z) & = A_{10}(z) \\
	\cos(z)  & = A_{20}(z) \\
	\sin(z)  & = A_{21}(z)
\end{align}
The goal of these notes is to present explicit results involving the $B_{pq}(z)$ and $A_{pq}(z)$ functions for other values of $p$ and $q$. In this way, you can generalize these elementary functions.
%%%%%%%%%%%%%%%%%%%%%%%%%%%%%%%%%%%%%%%%%%%%%%%%%%%%%%%%%%%%%%%%%%%%%%%%%%%%%%%%
\section{Pochhammer Symbol}
%%%%%%%%%%%%%%%%%%%%%%%%%%%%%%%%%%%%%%%%%%%%%%%%%%%%%%%%%%%%%%%%%%%%%%%%%%%%%%%%
The Pochhammer symbol is
\begin{equation}
	\left(z\right)_{n} = \frac{\Gamma(z + n)}{\Gamma(z)} = \prod_{k = 0}^{n-1} \left(z + k\right) = \prod_{l = 0}^{n-1} \left(n - 1 + z - l\right)
\end{equation}
Equivalently, you have
\begin{equation}
	\Gamma(z + n) = \left(z\right)_{n} \Gamma(z)
\end{equation}
This appears in the expression
\begin{equation}
	\Gamma\left(pn+q+1\right) = \left(pn+1\right)_{q} \Gamma\left(pn+1\right)
\end{equation}
Note that, for non-negative integers $p$ and $n$, you have
\begin{equation}
	\Gamma\left(pn+1\right) = \prod_{k=1}^{pn} k
\end{equation}
Using $k = pl + m$ with $0 \leq l \leq n - 1$ and $1 \leq m \leq p$, the above result can be written as a double product:
\begin{equation}
	\Gamma\left(pn+1\right) = \prod_{m = 1}^{p} \prod_{l = 0}^{n-1} \left(p l + m\right) = p^{pn} \prod_{m = 1}^{p} \prod_{l = 0}^{n-1} \left(l + \frac{m}{p}\right)
\end{equation}
That is
\begin{equation}
	\Gamma\left(pn+1\right) = p^{pn} \prod_{m = 1}^{p} \left(\frac{m}{p}\right)_{n}
\end{equation}
Hence
\begin{equation}
	\Gamma\left(pn+q+1\right) = p^{pn} \left(pn+1\right)_{q} \prod_{m = 1}^{p} \left(\frac{m}{p}\right)_{n}
\end{equation}
The product with $p$ terms can be split into two products: one with $q$ terms and one with $p-q$ terms:
\begin{equation}
	\Gamma\left(pn+1\right) = p^{pn} \left[\left(pn+1\right)_{q} \prod_{k = 1}^{q} \left(\frac{k}{p}\right)_{n}\right] \left[\prod_{m = q+1}^{p} \left(\frac{m}{p}\right)_{n}\right]
\end{equation}
Note that
\begin{equation}
	\left(pn+1\right)_{q} = \Gamma\left(q + 1\right) \prod_{k = 1}^{q} \left(\frac{p}{k}\right) \left(n + \frac{k}{p}\right)
\end{equation}
The Pochhammer symbol satisfies the following property:
\begin{equation}
	(z+1)_{n} = \left(\frac{z+n}{z}\right) \left(z\right)_{n}
\end{equation}
Thus
\begin{equation}
	\left(pn+1\right)_{q} \prod_{k = 1}^{q} \left(\frac{k}{p}\right)_{n} = \Gamma(q + 1) \prod_{k = 1}^{q} \left(1 + \frac{k}{p}\right)_{n}
\end{equation}
Hence
\begin{equation}
	\Gamma(pn+q+1) = p^{pn} \Gamma(q+1) \left[\prod_{k = 1}^{q} \left(1 + \frac{k}{p}\right)_{n}\right] \left[\prod_{m = q+1}^{p} \left(\frac{m}{p}\right)_{n}\right]
\end{equation}
Note that
\begin{equation}
    \prod_{k = 1}^{q} \left(1 + \frac{k}{p} \right)_{n} = \prod_{k = 1}^{q} \left(\frac{p + k}{p}\right)_{n} = \prod_{k=1}^{q} \left(\frac{p + q + 1 - k}{p}\right)_{n}
\end{equation}
and also
\begin{equation}
    \prod_{m = q+1}^{p} \left(\frac{m}{p}\right)_{n} = \prod_{k = 0}^{p-q-1} \left(\frac{p - k}{p}\right)_{n} = \prod_{k = q + 1}^{p} \left(\frac{p + q + 1 - k}{p}\right)_{n}
\end{equation}
All these lead to the final result:
\begin{equation}
    \Gamma(pn+q+1) = p^{pn} \Gamma(q+1) \prod_{k = 1}^{p} \left(\frac{p+q+1-k}{p}\right)_{n} = p^{pn} \Gamma(q+1) \prod_{l=1}^{p} \left(\frac{q+l}{p}\right)_{n}
\end{equation}
The second equality follows from using $l = p + 1 - k$.

For $p = 1$, you can only have $q = 0$:
\begin{equation}
	\Gamma(n+1) = \Gamma(1) (1)_{n}
\end{equation}
For $p = 2$, you can have $q = 0$ and $q = 1$:
\begin{align}
	\Gamma(2n+1) & = 2^{2n} \Gamma(1) \left(\frac{1}{2}\right)_{n} (1)_{n} \\
	\Gamma(2n+2) & = 2^{2n} \Gamma(2) (1)_{n} \left(\frac{3}{2}\right)_{n}
\end{align}
For $p = 3$, you can have $q = 0$, $q = 1$, and $q = 2$:
\begin{align}
	\Gamma(3n+1) & = 3^{3n} \Gamma(1) \left(\frac{1}{3}\right)_{n} \left(\frac{2}{3}\right)_{n} \left(1\right)_{n} \\
	\Gamma(3n+2) & = 3^{3n} \Gamma(2) \left(\frac{2}{3}\right)_{n} \left(1\right)_{n} \left(\frac{4}{3}\right)_{n} \\
	\Gamma(3n+3) & = 3^{3n} \Gamma(3) \left(1\right)_{n} \left(\frac{4}{3}\right)_{n} \left(\frac{5}{3}\right)_{n}
\end{align}
For $p = 4$, you can have $q = 0$, $q = 1$, $q = 2$, and $q = 3$:
\begin{align}
	\Gamma(4n+1) & = 4^{4n} \Gamma(1) \left(\frac{1}{4}\right)_{n} \left(\frac{1}{2}\right)_{n} \left(\frac{3}{4}\right)_{n} \left(1\right)_{n} \\
	\Gamma(4n+2) & = 4^{4n} \Gamma(2) \left(\frac{1}{2}\right)_{n} \left(\frac{3}{4}\right)_{n} \left(1\right)_{n} \left(\frac{5}{4}\right)_{n} \\
	\Gamma(4n+3) & = 4^{4n} \Gamma(3) \left(\frac{3}{4}\right)_{n} \left(1\right)_{n} \left(\frac{5}{4}\right)_{n} \left(\frac{3}{2}\right)_{n} \\
	\Gamma(4n+4) & = 4^{4n} \Gamma(4) \left(1\right)_{n} \left(\frac{5}{4}\right)_{n} \left(\frac{3}{2}\right)_{n} \left(\frac{7}{4}\right)_{n}
\end{align}
For $p = 5$, you can have $q = 0$, $q = 1$, $q = 2$, $q = 3$, and $q = 4$:
\begin{align}
	\Gamma(5n+1) & = 5^{5n} \Gamma(1) \left(\frac{1}{5}\right)_{n} \left(\frac{2}{5}\right)_{n} \left(\frac{3}{5}\right)_{n} \left(\frac{4}{5}\right)_{n} \left(1\right)_{n} \\
	\Gamma(5n+2) & = 5^{5n} \Gamma(2) \left(\frac{2}{5}\right)_{n} \left(\frac{3}{5}\right)_{n} \left(\frac{4}{5}\right)_{n} \left(1\right)_{n} \left(\frac{6}{5}\right)_{n} \\
	\Gamma(5n+3) & = 5^{5n} \Gamma(3) \left(\frac{3}{5}\right)_{n} \left(\frac{4}{5}\right)_{n} \left(1\right)_{n} \left(\frac{6}{5}\right)_{n} \left(\frac{7}{5}\right)_{n} \\
    \Gamma(5n+4) & = 5^{5n} \Gamma(4) \left(\frac{4}{5}\right)_{n} \left(1\right)_{n} \left(\frac{6}{5}\right)_{n} \left(\frac{7}{5}\right)_{n} \left(\frac{8}{5}\right)_{n} \\
    \Gamma(5n+5) & = 5^{5n} \Gamma(5) \left(1\right)_{n} \left(\frac{6}{5}\right)_{n} \left(\frac{7}{5}\right)_{n} \left(\frac{8}{5}\right)_{n} \left(\frac{9}{5}\right)_{n}
\end{align}
For $p = 6$, you can have $q = 0$, $q = 1$, $q = 2$, $q = 3$, $q = 4$, and $q = 5$:
\begin{align}
	\Gamma(6n+1) & = 6^{6n} \Gamma(1) \left(\frac{1}{6}\right)_{n} \left(\frac{1}{3}\right)_{n} \left(\frac{1}{2}\right)_{n} \left(\frac{2}{3}\right)_{n} \left(\frac{5}{6}\right)_{n} \left(1\right)_{n} \\
	\Gamma(6n+2) & = 6^{6n} \Gamma(2) \left(\frac{1}{3}\right)_{n} \left(\frac{1}{2}\right)_{n} \left(\frac{2}{3}\right)_{n} \left(\frac{5}{6}\right)_{n} \left(1\right)_{n} \left(\frac{7}{6}\right)_{n} \\
	\Gamma(6n+3) & = 6^{6n} \Gamma(3) \left(\frac{1}{2}\right)_{n} \left(\frac{2}{3}\right)_{n} \left(\frac{5}{6}\right)_{n} \left(1\right)_{n} \left(\frac{7}{6}\right)_{n} \left(\frac{4}{3}\right)_{n} \\
    \Gamma(6n+4) & = 6^{6n} \Gamma(4) \left(\frac{2}{3}\right)_{n} \left(\frac{5}{6}\right)_{n} \left(1\right)_{n} \left(\frac{7}{6}\right)_{n} \left(\frac{4}{3}\right)_{n} \left(\frac{3}{2}\right)_{n} \\
    \Gamma(6n+5) & = 6^{6n} \Gamma(5) \left(\frac{5}{6}\right)_{n} \left(1\right)_{n} \left(\frac{7}{6}\right)_{n} \left(\frac{4}{3}\right)_{n} \left(\frac{3}{2}\right)_{n} \left(\frac{5}{3}\right)_{n} \\
    \Gamma(6n+6) & = 6^{6n} \Gamma(6) \left(1\right)_{n} \left(\frac{7}{6}\right)_{n} \left(\frac{4}{3}\right)_{n} \left(\frac{3}{2}\right)_{n} \left(\frac{5}{3}\right)_{n} \left(\frac{11}{6}\right)_{n}
\end{align}
%%%%%%%%%%%%%%%%%%%%%%%%%%%%%%%%%%%%%%%%%%%%%%%%%%%%%%%%%%%%%%%%%%%%%%%%%%%%%%%%
\section{Generalized Hypergeometric Function}
%%%%%%%%%%%%%%%%%%%%%%%%%%%%%%%%%%%%%%%%%%%%%%%%%%%%%%%%%%%%%%%%%%%%%%%%%%%%%%%%
The ${}_{P}F_{Q}$ generalized hypergeometric function is defined as
\begin{equation}
    {}_{P}F_{Q}(a_{1}, a_{2}, ..., a_{P}; b_{1}, b_{2}, ..., b_{Q}; z) \equiv \sum_{n=0}^{\infty} \frac{(a_{1})_{n} (a_{2})_{n} \cdots (a_{P})_{n}}{(b_{1})_{n} (b_{2})_{n} \cdots (b_{Q})_{n}} \frac{1}{\Gamma(n+1)} z^{n}
\end{equation}
The relevant case involves $P = 0$:
\begin{equation}
    {}_{0}F_{Q}(b_{1}, b_{2}, ..., b_{Q}; z) = \sum_{n=0}^{\infty} \frac{1}{(b_{1})_{n} (b_{2})_{n} \cdots (b_{Q})_{n}} \frac{1}{\Gamma(n+1)} z^{n}
\end{equation}
Since
\begin{equation}
    \Gamma(n+1) = (1)_{n}
\end{equation}
it is convenient to define a different generalized hypergeometric function:
\begin{equation}
    G_{m}(b_{1}, b_{2}, ..., b_{m}; z) \equiv \sum_{n=0}^{\infty} \frac{1}{(b_{1})_{n} (b_{2})_{n} \cdots (b_{m})_{n}} z^{n}
\end{equation}
The traditional generalized hypergeometric function is found via
\begin{equation}
    {}_{0}F_{Q}(b_{1}, b_{2}, ..., b_{Q}; z) = G_{Q+1}(1, b_{1}, b_{2}, ..., b_{Q}; z)
\end{equation}
Recall that
\begin{equation}
    B_{pq}(z) = \sum_{n=0}^{\infty} \frac{1}{\Gamma(pn+q+1)} z^{pn+q}
\end{equation}
Using the above result for $\Gamma(pn+q+1)$ gives
\begin{equation}
    B_{pq}(z) = \frac{z^{q}}{\Gamma(q+1)} G_{p}\left(\frac{q + p}{p}, \frac{q + p-1}{p}, \frac{q + p-2}{p}, ..., \frac{q + 2}{p}, \frac{q+1}{p}; \frac{z^{p}}{p^{p}}\right)
\end{equation}
In terms of $G_{m}$ or ${}_{0}F_{m}$, you have
\begin{align}
    B_{10}(z) &= G_{1}(1; z) = {}_{0} F_{0}(z) = \exp(z) \\
    B_{20}(z) &= G_{2}\left(\frac{1}{2}, 1; \frac{z^{2}}{2^{2}}\right) = {}_{0} F_{1}\left(\frac{1}{2}; \frac{z^{2}}{2^{2}}\right) = \cosh(z) \\
    \frac{B_{21}(z)}{z} &= G_{2}\left(1, \frac{3}{2}; \frac{z^{2}}{2^{2}}\right) = {}_{0} F_{1}\left(\frac{3}{2}; \frac{z^{2}}{2^{2}}\right) = \frac{\sinh(z)}{z} \\
    B_{30}(z) &= G_{3}\left(\frac{1}{3}, \frac{2}{3}, 1; \frac{z^{3}}{3^{3}}\right) = {}_{0}F_{2}\left(\frac{1}{3}, \frac{2}{3}; \frac{z^{3}}{3^{3}}\right) \\
    B_{40}(z) &= G_{4}\left(\frac{1}{4}, \frac{1}{2}, \frac{3}{4}, 1; \frac{z^{4}}{4^{4}}\right) = {}_{0}F_{3}\left(\frac{1}{4}, \frac{1}{2}, \frac{3}{4}; \frac{z^{4}}{4^{4}}\right)
\end{align}
and so on.