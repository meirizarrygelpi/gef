% Copyright 2018 Melvin Eloy Irizarry-Gelpí
\chapter{Definitions}
%%%%%%%%%%%%%%%%%%%%%%%%%%%%%%%%%%%%%%%%%%%%%%%%%%%%%%%%%%%%%%%%%%%%%%%%%%%%%%%%
Here are some definitions.
%%%%%%%%%%%%%%%%%%%%%%%%%%%%%%%%%%%%%%%%%%%%%%%%%%%%%%%%%%%%%%%%%%%%%%%%%%%%%%%%
\section{Euler Gamma Function}
%%%%%%%%%%%%%%%%%%%%%%%%%%%%%%%%%%%%%%%%%%%%%%%%%%%%%%%%%%%%%%%%%%%%%%%%%%%%%%%%
The \textbf{Euler Gamma function} satisfies the property
\begin{equation}
    \Gamma(z + 1) = z \Gamma(z)
\end{equation}
This expression can be rearranged into
\begin{equation}
    \frac{1}{\Gamma(z)} = \frac{z}{\Gamma(z + 1)}
\end{equation}
This identity will be used extensively later.
%%%%%%%%%%%%%%%%%%%%%%%%%%%%%%%%%%%%%%%%%%%%%%%%%%%%%%%%%%%%%%%%%%%%%%%%%%%%%%%%
\section{Exponential Function}
%%%%%%%%%%%%%%%%%%%%%%%%%%%%%%%%%%%%%%%%%%%%%%%%%%%%%%%%%%%%%%%%%%%%%%%%%%%%%%%%
The \textbf{exponential function} can be defined as a power series:
\begin{equation}
    \exp(z) = \sum_{n = 0}^{\infty} \frac{1}{\Gamma(n + 1)} z^{n}
\end{equation}
Using the above property of the Euler Gamma function, you find the differentiation formula:
\begin{equation}
    \partial \exp(z) = \exp(z)
\end{equation}
This expression can be rearranged into
\begin{equation}
    \left( \partial - 1 \right) \exp(z) = 0
\end{equation}
This is a first-order differential equation; there is only one linearly-independent solution.

The \textbf{reciprocal of the exponential function} can be defined as an \textbf{alternating} power series:
\begin{equation}
    \exp({-z}) = \sum_{n = 0}^{\infty} \frac{(-1)^{n}}{\Gamma(n + 1)} z^{n}
\end{equation}
Using the above property of the Euler Gamma function, you find the differentiation formula:
\begin{equation}
    \partial \exp({-z}) = -\exp({-z})
\end{equation}
This expression can be rearranged into
\begin{equation}
    \left( \partial + 1 \right) \exp({-z}) = 0
\end{equation}
This is a first-order differential equation; there is only one linearly-independent solution.
%%%%%%%%%%%%%%%%%%%%%%%%%%%%%%%%%%%%%%%%%%%%%%%%%%%%%%%%%%%%%%%%%%%%%%%%%%%%%%%%
\section{Hyperbolic Functions}
%%%%%%%%%%%%%%%%%%%%%%%%%%%%%%%%%%%%%%%%%%%%%%%%%%%%%%%%%%%%%%%%%%%%%%%%%%%%%%%%
The two basic \textbf{hyperbolic functions} can be defined as power series:
\begin{align}
    \cosh(z) &= \sum_{n = 0}^{\infty} \frac{1}{\Gamma(2n + 1)} z^{2n} & \sinh(z) &= \sum_{n = 0}^{\infty} \frac{1}{\Gamma(2n + 2)} z^{2n + 1}
\end{align}
Taking one derivative gives the differentiation formulae:
\begin{align}
    \partial \sinh(z) &= \cosh(z) & \partial \cosh(z) &= \sinh(z)
\end{align}
Taking two derivatives gives
\begin{align}
    \partial^{2} \cosh(z) &= \cosh(z) & \partial^{2} \sinh(z) &= \sinh(z)
\end{align}
These expressions can be rearranged into
\begin{align}
    \left( \partial^{2} - 1 \right) \cosh(z) &= 0 & \left( \partial^{2} - 1 \right) \sinh(z) &= 0
\end{align}
These are second-order differential equations. Indeed, the most general solution to the differential equation
\begin{equation}
    \left( \partial^{2} - 1 \right) f(z) = 0
\end{equation}
is
\begin{equation}
    f(z) = b_{0} \cosh(z) + b_{1} \sinh(z)
\end{equation}
These two hyperbolic functions are examples of elementary functions.
%%%%%%%%%%%%%%%%%%%%%%%%%%%%%%%%%%%%%%%%%%%%%%%%%%%%%%%%%%%%%%%%%%%%%%%%%%%%%%%%
\section{Trigonometric Functions}
%%%%%%%%%%%%%%%%%%%%%%%%%%%%%%%%%%%%%%%%%%%%%%%%%%%%%%%%%%%%%%%%%%%%%%%%%%%%%%%%
The two basic \textbf{trigonometric functions} can be defined as alternating power series:
\begin{align}
    \cos(z) &= \sum_{n = 0}^{\infty} \frac{(-1)^{n}}{\Gamma(2n + 1)} z^{2n} & \sin(z) &= \sum_{n = 0}^{\infty} \frac{(-1)^{n}}{\Gamma(2n + 2)} z^{2n + 1}
\end{align}
Taking one derivative gives the differentiation formulae:
\begin{align}
    \partial \cos(z) &= -\sin(z) & \partial \sin(z) &= \cos(z)
\end{align}
Taking two derivatives gives
\begin{align}
    \partial^{2} \cos(z) &= -\cos(z) & \partial^{2} \sin(z) &= -\sin(z)
\end{align}
These expressions can be rearranged into
\begin{align}
    \left( \partial^{2} + 1 \right) \cos(z) &= 0 & \left( \partial^{2} + 1 \right) \sin(z) &= 0
\end{align}
These are second-order differential equations. Indeed, the most general solution to the differential equation
\begin{equation}
    \left( \partial^{2} + 1 \right) f(z) = 0
\end{equation}
is
\begin{equation}
    f(z) = a_{0} \cos(z) + a_{1} \sin(z)
\end{equation}
These two trigonometric functions are also examples of elementary functions.
%%%%%%%%%%%%%%%%%%%%%%%%%%%%%%%%%%%%%%%%%%%%%%%%%%%%%%%%%%%%%%%%%%%%%%%%%%%%%%%%
\section{Mittag-Leffler Function}
%%%%%%%%%%%%%%%%%%%%%%%%%%%%%%%%%%%%%%%%%%%%%%%%%%%%%%%%%%%%%%%%%%%%%%%%%%%%%%%%
The \textbf{Mittag-Leffler function} is defined as the power series
\begin{equation}
    \operatorname{ML}(\alpha, \beta; z) \equiv \sum_{n = 0}^{\infty} \frac{1}{\Gamma(\alpha n + \beta)} z^{n}
\end{equation}
All the elementary functions introduced above are special cases of the ML function:
\begin{align}
    \exp(z) &= \operatorname{ML}(1, 1; z) \\
    \cosh(z) &= \operatorname{ML}\left(2, 1; z^{2}\right) \\
    \sinh(z) &= \operatorname{ML}\left(2, 2; z^{2}\right) \\
    \exp(-z) &= \operatorname{ML}(1, 1; -z) \\
    \cos(z) &= \operatorname{ML}\left(2, 1; -z^{2}\right) \\
    \sin(z) &= \operatorname{ML}\left(2, 2; -z^{2}\right)
\end{align}
The ML function satisfies many relations. For example, multiplication by the argument variable gives:
\begin{equation}
    z \operatorname{ML}(\alpha, \beta; z) = \operatorname{ML}(\alpha, \beta - \alpha; z) - \frac{1}{\Gamma(\beta - \alpha)}
\end{equation}
This expression can be rearranged into
\begin{equation}
    \operatorname{ML}(\alpha, \beta - \alpha; z) = z \operatorname{ML}(\alpha, \beta; z) + \frac{1}{\Gamma(\beta - \alpha)}
\end{equation}
In a similar way, for $k \geq 1$ you get
\begin{equation}
    z^{k} \operatorname{ML}(\alpha, \beta; z) = \operatorname{ML}(\alpha, \beta - k\alpha; z) - \sum_{n = 0}^{k-1} \frac{1}{\Gamma(\alpha n + \beta - k \alpha)} z^{n}
\end{equation}
Division by the argument variable gives another set of relations:
\begin{equation}
    \frac{1}{z} \operatorname{ML}(\alpha, \beta; z) = \operatorname{ML}(\alpha, \beta + \alpha; z) + \frac{1}{z \Gamma(\beta)}
\end{equation}
This expression can be rearranged into
\begin{equation}
    \operatorname{ML}(\alpha, \beta; z) = z \operatorname{ML}(\alpha, \beta + \alpha; z) + \frac{1}{\Gamma(\beta)}
\end{equation}
In a similar way, for $k \geq 1$ you get
\begin{equation}
    \frac{1}{z^{k}} \operatorname{ML}(\alpha, \beta; z) = \operatorname{ML}(\alpha, \beta + k \alpha; z) + \frac{1}{z^{k}} \sum_{n = 0}^{k - 1} \frac{1}{\Gamma(\alpha n + \beta)} z^{n}
\end{equation}
Or equivalently
\begin{equation}
    \operatorname{ML}(\alpha, \beta; z) = z^{k} \operatorname{ML}(\alpha, \beta + k \alpha; z) + \sum_{n = 0}^{k - 1} \frac{1}{\Gamma(\alpha n + \beta)} z^{n}
\end{equation}
%%%%%%%%%%%%%%%%%%%%%%%%%%%%%%%%%%%%%%%%%%%%%%%%%%%%%%%%%%%%%%%%%%%%%%%%%%%%%%%%
\section{Generalized Mittag-Leffler Functions}
%%%%%%%%%%%%%%%%%%%%%%%%%%%%%%%%%%%%%%%%%%%%%%%%%%%%%%%%%%%%%%%%%%%%%%%%%%%%%%%%
The \textbf{generalized Mittag-Leffler functions} are defined as the power series:
\begin{align}
    \operatorname{GML}(a, b, c, d; z) &\equiv \sum_{n = 0}^{\infty} \frac{1}{\Gamma(a n + b + 1)} z^{cn+d} \\
    \operatorname{AML}(a, b, c, d; z) &\equiv \sum_{n = 0}^{\infty} \frac{(-1)^{n}}{\Gamma(a n + b + 1)} z^{cn+d}
\end{align}
In terms of GML, the ML function can be written as
\begin{equation}
    \operatorname{ML}(\alpha, \beta; z) = \operatorname{GML}(\alpha, \beta - 1, 1, 0; z)
\end{equation}
Conversely, in terms of ML, the GML and AML functions can be written as
\begin{align}
    \operatorname{GML}(a, b, c, d; z) &= z^{d} \operatorname{ML}\left(a, b+1; z^{c}\right) \\
    \operatorname{AML}(a, b, c, d; z) &= z^{d} \operatorname{ML}\left(a, b+1; -z^{c}\right)
\end{align}
In terms of GML and AML, the elementary functions above are given by
\begin{align}
    \exp(z) &= \operatorname{GML}(1, 0, 1, 0; z) \\
    \cosh(z) &= \operatorname{GML}(2, 0, 1, 0; z) \\
    \sinh(z) &= \operatorname{GML}(2, 1, 2, 1; z) \\
    \exp(-z) &= \operatorname{AML}(1, 0, 1, 0; z) \\
    \cos(z) &= \operatorname{AML}(2, 0, 1, 0; z) \\
    \sin(z) &= \operatorname{AML}(2, 1, 2, 1; z)
\end{align}
For simplicity, define the following power series:
\begin{align}
    B_{pq}(z) &\equiv \operatorname{GML}(p, q, p, q; z) = \sum_{n = 0}^{\infty} \frac{1}{\Gamma(p n + q + 1)} z^{pn + q} \\
    A_{pq}(z) &\equiv \operatorname{AML}(p, q, p, q; z) = \sum_{n = 0}^{\infty} \frac{(-1)^{n}}{\Gamma(p n + q + 1)} z^{pn + q}
\end{align}
These functions are solutions to $p$-order differential equations:
\begin{align}
    \left( \partial^{p} - 1 \right) B_{pq}(z) &= 0 & \left( \partial^{p} + 1 \right) A_{pq}(z) &= 0 & q &= 0, 1, ..., p -1
\end{align}
Indeed, if
\begin{align}
    f(z) &= \sum_{q = 0}^{p-1} b_{q} B_{pq}(z) \\
    g(z) &= \sum_{q = 0}^{p-1} a_{q} A_{pq}(z)
\end{align}
then
\begin{align}
    \left( \partial^{p} - 1 \right) f(z) &= 0 \\
    \left( \partial^{p} + 1 \right) g(z) &= 0
\end{align}
In terms of these functions, you have
\begin{align}
    \exp(z) &= B_{10}(z) \\
    \cosh(z) &= B_{20}(z) \\
    \sinh(z) &= B_{21}(z) \\
    \exp(-z) &= A_{10}(z) \\
    \cos(z) &= A_{20}(z) \\
    \sin(z) &= A_{21}(z)
\end{align}
The goal of these notes is to present explicit results involving the $B_{pq}(z)$ and $A_{pq}(z)$ functions for other values of $p$ and $q$. In this way, you can generalize these elementary functions.