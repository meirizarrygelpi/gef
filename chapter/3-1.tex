% Copyright 2018 Melvin Eloy Irizarry-Gelpí
\chapter{1-Exponential Functions}
%%%%%%%%%%%%%%%%%%%%%%%%%%%%%%%%%%%%%%%%%%%%%%%%%%%%%%%%%%%%%%%%%%%%%%%%%%%%%%%%
The 1-exponential function is among the simplest of the traditional elementary functions.
%%%%%%%%%%%%%%%%%%%%%%%%%%%%%%%%%%%%%%%%%%%%%%%%%%%%%%%%%%%%%%%%%%%%%%%%%%%%%%%%
\section{Definition}
%%%%%%%%%%%%%%%%%%%%%%%%%%%%%%%%%%%%%%%%%%%%%%%%%%%%%%%%%%%%%%%%%%%%%%%%%%%%%%%%
The 1-exponential function is
\begin{equation}
    B_{10}(z) = \sum_{n = 0}^{\infty} \frac{1}{\Gamma(n+1)} z^{n} = \exp(z)
\end{equation}
%%%%%%%%%%%%%%%%%%%%%%%%%%%%%%%%%%%%%%%%%%%%%%%%%%%%%%%%%%%%%%%%%%%%%%%%%%%%%%%%
\section{Identities}
%%%%%%%%%%%%%%%%%%%%%%%%%%%%%%%%%%%%%%%%%%%%%%%%%%%%%%%%%%%%%%%%%%%%%%%%%%%%%%%%
Here are some important identities.
%%%%%%%%%%%%%%%%%%%%%%%%%%%%%%%%%%%%%%%%%%%%%%%%%%%%%%%%%%%%%%%%%%%%%%%%%%%%%%%%
\subsection{Addition Formula}
%%%%%%%%%%%%%%%%%%%%%%%%%%%%%%%%%%%%%%%%%%%%%%%%%%%%%%%%%%%%%%%%%%%%%%%%%%%%%%%%
The addition formula is
\begin{equation}
    B_{10}(w + z) = B_{10}(w) B_{10}(z)
\end{equation}
%%%%%%%%%%%%%%%%%%%%%%%%%%%%%%%%%%%%%%%%%%%%%%%%%%%%%%%%%%%%%%%%%%%%%%%%%%%%%%%%
\subsection{de Moivre Scaling Formula}
%%%%%%%%%%%%%%%%%%%%%%%%%%%%%%%%%%%%%%%%%%%%%%%%%%%%%%%%%%%%%%%%%%%%%%%%%%%%%%%%
For the 1-exponential case, the de Moivre scale factor is
\begin{equation}
    r_{21} = -1
\end{equation}
This leads to the following de Moivre scaling formula:
\begin{equation}
    B_{10}(-z) = \sum_{n = 0}^{\infty} \frac{(-1)^{n}}{\Gamma(n + 1)} z^{n} = A_{10}(z)
\end{equation}
Thus, a de Moivre scaling of $B_{10}$ gives the alternating function $A_{10}$.
%%%%%%%%%%%%%%%%%%%%%%%%%%%%%%%%%%%%%%%%%%%%%%%%%%%%%%%%%%%%%%%%%%%%%%%%%%%%%%%%
\subsection{Differentiation Formula}
%%%%%%%%%%%%%%%%%%%%%%%%%%%%%%%%%%%%%%%%%%%%%%%%%%%%%%%%%%%%%%%%%%%%%%%%%%%%%%%%
The differentiation formula is
\begin{equation}
    \partial B_{10} = B_{10}
\end{equation}
%%%%%%%%%%%%%%%%%%%%%%%%%%%%%%%%%%%%%%%%%%%%%%%%%%%%%%%%%%%%%%%%%%%%%%%%%%%%%%%%
\subsection{Algebraic Formula}
%%%%%%%%%%%%%%%%%%%%%%%%%%%%%%%%%%%%%%%%%%%%%%%%%%%%%%%%%%%%%%%%%%%%%%%%%%%%%%%%
Recall that
\begin{equation}
    r_{20} + r_{21} = 0
\end{equation}
Since
\begin{equation}
    \exp\left(r_{20} z\right) \exp\left(r_{21} z\right) = 1
\end{equation}
you also have
\begin{equation}
    B_{10}(z) A_{10}(z) = 1
\end{equation}
%%%%%%%%%%%%%%%%%%%%%%%%%%%%%%%%%%%%%%%%%%%%%%%%%%%%%%%%%%%%%%%%%%%%%%%%%%%%%%%%
\section{Properties}
%%%%%%%%%%%%%%%%%%%%%%%%%%%%%%%%%%%%%%%%%%%%%%%%%%%%%%%%%%%%%%%%%%%%%%%%%%%%%%%%
Here are some important properties.
%%%%%%%%%%%%%%%%%%%%%%%%%%%%%%%%%%%%%%%%%%%%%%%%%%%%%%%%%%%%%%%%%%%%%%%%%%%%%%%%
\subsection{Special Values}
%%%%%%%%%%%%%%%%%%%%%%%%%%%%%%%%%%%%%%%%%%%%%%%%%%%%%%%%%%%%%%%%%%%%%%%%%%%%%%%%
Note that
\begin{equation}
    B_{10}(0) = 1
\end{equation}
%%%%%%%%%%%%%%%%%%%%%%%%%%%%%%%%%%%%%%%%%%%%%%%%%%%%%%%%%%%%%%%%%%%%%%%%%%%%%%%%
\subsection{Real and Imaginary Parts}
%%%%%%%%%%%%%%%%%%%%%%%%%%%%%%%%%%%%%%%%%%%%%%%%%%%%%%%%%%%%%%%%%%%%%%%%%%%%%%%%
Let $x$ and $y$ be real variables, and $z = x + y i$ be a complex variable. Then
\begin{equation}
    B_{10}(z) = \exp({x}) \left[ \cos(y) + i \sin(y) \right]
\end{equation}
Thus, the real part of the 1-exponential function is
\begin{equation}
    \Re\left[ B_{10}(x + y i) \right] = \exp({x}) \cos(y)
\end{equation}
and the imaginary part of the 1-exponential function is
\begin{equation}
    \Im\left[ B_{10}(x + y i) \right] = \exp({x}) \sin(y)
\end{equation}
%%%%%%%%%%%%%%%%%%%%%%%%%%%%%%%%%%%%%%%%%%%%%%%%%%%%%%%%%%%%%%%%%%%%%%%%%%%%%%%%
\subsection{Anti-Periodicity}
%%%%%%%%%%%%%%%%%%%%%%%%%%%%%%%%%%%%%%%%%%%%%%%%%%%%%%%%%%%%%%%%%%%%%%%%%%%%%%%%
Let $u$ and $v$ be real constants, and $x$ and $y$ be real variables. A function $f$ is anti-periodic with anti-period $u + v i $ if
\begin{equation}
    f(x + y i + u + v i) + f(x + y i) = 0
\end{equation}
for all values of $x$ and $y$.

For $B_{10}$ you have
\begin{align}
    \Re\left[ B_{10}(x + y i + u + v i) \right] &= \exp({x} + u) \cos(y + v) \\
    \Im\left[ B_{10}(x + y i + u + v i) \right] &= \exp({x} + u) \sin(y + v)
\end{align}
In both cases, anti-periodicity requires $u = 0$ and $v = \pi$. That is,
\begin{equation}
    B_{10}(z + \pi i) + B_{10}(z) = 0
\end{equation}
%%%%%%%%%%%%%%%%%%%%%%%%%%%%%%%%%%%%%%%%%%%%%%%%%%%%%%%%%%%%%%%%%%%%%%%%%%%%%%%%
\subsection{Periodicity}
%%%%%%%%%%%%%%%%%%%%%%%%%%%%%%%%%%%%%%%%%%%%%%%%%%%%%%%%%%%%%%%%%%%%%%%%%%%%%%%%
Let $u$ and $v$ be real constants, and $x$ and $y$ be real variables. A function $f$ is periodic with period $u + v i $ if
\begin{equation}
    f(x + y i + u + v i) - f(x + y i) = 0
\end{equation}
for all values of $x$ and $y$.

Since $B_{10}$ is anti-periodic with anti-period $\pi i$, this function is also periodic with period $2 \pi i$. That is,
\begin{equation}
    B_{10}(z + 2 \pi i) - B_{10}(z) = 0
\end{equation}
%%%%%%%%%%%%%%%%%%%%%%%%%%%%%%%%%%%%%%%%%%%%%%%%%%%%%%%%%%%%%%%%%%%%%%%%%%%%%%%%
\subsection{Rational Phase Shift}
%%%%%%%%%%%%%%%%%%%%%%%%%%%%%%%%%%%%%%%%%%%%%%%%%%%%%%%%%%%%%%%%%%%%%%%%%%%%%%%%
Both anti-periodicity and periodicity are special cases of the rational phase shift. That is, let $m$ and $n$ be integers with $n \neq 0$. Then
\begin{equation}
    B_{10}\left(z + \frac{2\pi m i}{n}\right) = \exp\left( \frac{2\pi m i}{n} \right) B_{10}(z)
\end{equation}
Shifting the argument variable by a rational multiple of $2 \pi i$ gives the 1-exponential function up to a pure phase factor.