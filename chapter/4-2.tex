% Copyright 2018 Melvin Eloy Irizarry-Gelpí
\chapter{2-Hyperbolic Functions}
%%%%%%%%%%%%%%%%%%%%%%%%%%%%%%%%%%%%%%%%%%%%%%%%%%%%%%%%%%%%%%%%%%%%%%%%%%%%%%%%
The 2-hyperbolic functions are the familiar hyperbolic cosine and sine functions.
%%%%%%%%%%%%%%%%%%%%%%%%%%%%%%%%%%%%%%%%%%%%%%%%%%%%%%%%%%%%%%%%%%%%%%%%%%%%%%%%
\section{Definition}
%%%%%%%%%%%%%%%%%%%%%%%%%%%%%%%%%%%%%%%%%%%%%%%%%%%%%%%%%%%%%%%%%%%%%%%%%%%%%%%%
The 2-hyperbolic functions are
\begin{align}
    B_{20}(z) &= \sum_{n = 0}^{\infty} \frac{1}{\Gamma\left(2n+1\right)} z^{2n} \\
    B_{21}(z) &= \sum_{n = 0}^{\infty} \frac{1}{\Gamma\left(2n+2\right)} z^{2n+1}
\end{align}
These two functions are the familiar hyperbolic cosine and sine functions. Here is a (long winded) way to establish the relation between these functions and the exponential function. First, consider the following relations:
\begin{align}
    \exp\left( z \right) &= B_{20}(z) + B_{21}(z) \\
    \exp\left( -z \right) &= B_{20}(z) - B_{21}(z)
\end{align}
Here the $r_{mn}$ are defined in (\ref{eq.2.r4}). This system of equations can be stated as a matrix equation:
\begin{equation}
    \begin{bmatrix}
        \exp\left( z \right) \\ \exp\left( -z \right)
    \end{bmatrix} = 
    \begin{bmatrix}
        1 & 1 \\ 1 & -1
    \end{bmatrix}
    \begin{bmatrix}
        B_{20}(z) \\ B_{21}(z)
    \end{bmatrix}
\end{equation}
You can recognize the matrix $E_{2}$ from (\ref{eq.2.E2}). Inverting this matrix gives
\begin{equation}
    \begin{bmatrix}
        B_{20}(z) \\ B_{21}(z)
    \end{bmatrix} = \frac{1}{2}
    \begin{bmatrix}
        1 & 1 \\ 1 & -1
    \end{bmatrix}
    \begin{bmatrix}
        \exp\left( z \right) \\ \exp\left( -z \right)
    \end{bmatrix}
\end{equation}
You can recognize the matrix $F_{2}$ from (\ref{eq.2.F2}).

Explicitly, you have
\begin{align}
    B_{20}(z) &= \frac{1}{2} \left[ \exp(z) + \exp(-z) \right] = \cosh(z) \\
    B_{21}(z) &= \frac{1}{2} \left[ \exp(z) - \exp(-z) \right] = \sinh(z)
\end{align}
%%%%%%%%%%%%%%%%%%%%%%%%%%%%%%%%%%%%%%%%%%%%%%%%%%%%%%%%%%%%%%%%%%%%%%%%%%%%%%%%
\section{Properties}
%%%%%%%%%%%%%%%%%%%%%%%%%%%%%%%%%%%%%%%%%%%%%%%%%%%%%%%%%%%%%%%%%%%%%%%%%%%%%%%%
Here are some important properties.
%%%%%%%%%%%%%%%%%%%%%%%%%%%%%%%%%%%%%%%%%%%%%%%%%%%%%%%%%%%%%%%%%%%%%%%%%%%%%%%%
\subsection{Special Values}
%%%%%%%%%%%%%%%%%%%%%%%%%%%%%%%%%%%%%%%%%%%%%%%%%%%%%%%%%%%%%%%%%%%%%%%%%%%%%%%%
Note that
\begin{align}
    B_{20}(z) &= 1 & B_{21}(z) &= 0
\end{align}
%%%%%%%%%%%%%%%%%%%%%%%%%%%%%%%%%%%%%%%%%%%%%%%%%%%%%%%%%%%%%%%%%%%%%%%%%%%%%%%%
\subsection{Real and Imaginary Parts}
%%%%%%%%%%%%%%%%%%%%%%%%%%%%%%%%%%%%%%%%%%%%%%%%%%%%%%%%%%%%%%%%%%%%%%%%%%%%%%%%
Let $x$ and $y$ be real variables. The real and imaginary parts of $B_{20}$ are
\begin{align}
    \Re\left[ B_{20}(x + yi) \right] &= \cosh(x) \cos(y) \\
    \Im\left[ B_{20}(x + yi) \right] &= \sinh(x) \sin(y)
\end{align}
The real and imaginary parts of $B_{21}$ are
\begin{align}
    \Re\left[ B_{21}(x + yi) \right] &= \sinh(x) \cos(y) \\
    \Im\left[ B_{21}(x + yi) \right] &= \cosh(x) \sin(y)
\end{align}
%%%%%%%%%%%%%%%%%%%%%%%%%%%%%%%%%%%%%%%%%%%%%%%%%%%%%%%%%%%%%%%%%%%%%%%%%%%%%%%%
\subsection{Anti-Periodicity}
%%%%%%%%%%%%%%%%%%%%%%%%%%%%%%%%%%%%%%%%%%%%%%%%%%%%%%%%%%%%%%%%%%%%%%%%%%%%%%%%
Both $B_{20}$ and $B_{21}$ are anti-periodic:
\begin{align}
    B_{20}(z + \pi i) + B_{20}(z) &= 0 & B_{21}(z + \pi i) + B_{21}(z) &= 0
\end{align}
The anti-period is $\pi i$.
%%%%%%%%%%%%%%%%%%%%%%%%%%%%%%%%%%%%%%%%%%%%%%%%%%%%%%%%%%%%%%%%%%%%%%%%%%%%%%%%
\subsection{Periodicity}
%%%%%%%%%%%%%%%%%%%%%%%%%%%%%%%%%%%%%%%%%%%%%%%%%%%%%%%%%%%%%%%%%%%%%%%%%%%%%%%%
Both $B_{20}$ and $B_{21}$ are periodic:
\begin{align}
    B_{20}(z + 2\pi i) - B_{20}(z) &= 0 & B_{21}(z + 2\pi i) - B_{21}(z) &= 0
\end{align}
The period is $2\pi i$.
%%%%%%%%%%%%%%%%%%%%%%%%%%%%%%%%%%%%%%%%%%%%%%%%%%%%%%%%%%%%%%%%%%%%%%%%%%%%%%%%
\section{Identities}
%%%%%%%%%%%%%%%%%%%%%%%%%%%%%%%%%%%%%%%%%%%%%%%%%%%%%%%%%%%%%%%%%%%%%%%%%%%%%%%%
Here are some important identities.
%%%%%%%%%%%%%%%%%%%%%%%%%%%%%%%%%%%%%%%%%%%%%%%%%%%%%%%%%%%%%%%%%%%%%%%%%%%%%%%%
\subsection{Addition Formulae}
%%%%%%%%%%%%%%%%%%%%%%%%%%%%%%%%%%%%%%%%%%%%%%%%%%%%%%%%%%%%%%%%%%%%%%%%%%%%%%%%
The addition formulae are
\begin{align}
    B_{20}(x + y) &= B_{20}(x) B_{20}(y) + B_{21}(x) B_{21}(y) \\
    B_{21}(x + y) &= B_{20}(x) B_{21}(y) + B_{21}(x) B_{20}(y)
\end{align}
These formulae can be found as follows. Let $V$ be a square matrix such that
\begin{equation}
    V^{2} = I
\end{equation}
with $I$ the identity matrix. Then
\begin{align}
    \exp\left( z V \right) &= \sum_{n = 0}^{\infty} \frac{1}{\Gamma(2n + 1)} z^{2n} V^{2n} + \sum_{n = 0}^{\infty} \frac{1}{\Gamma(2n + 2)} z^{2n+1} V^{2n+1} \\ &= I B_{20}(z) + V B_{21}(z) \\
    \exp\left( -z V \right) &= \sum_{n = 0}^{\infty} \frac{1}{\Gamma(2n + 1)} z^{2n} V^{2n} - \sum_{n = 0}^{\infty} \frac{1}{\Gamma(2n + 2)} z^{2n+1} V^{2n+1} \\ &= I B_{20}(z) - V B_{21}(z)
\end{align}
Using the identity
\begin{equation}
    \exp\left( x V \right) \exp\left( y V \right) = \exp\left( xV + yV \right)
\end{equation}
you can find the above formulae.
%%%%%%%%%%%%%%%%%%%%%%%%%%%%%%%%%%%%%%%%%%%%%%%%%%%%%%%%%%%%%%%%%%%%%%%%%%%%%%%%
\subsection{de Moivre Scaling Formulae}
%%%%%%%%%%%%%%%%%%%%%%%%%%%%%%%%%%%%%%%%%%%%%%%%%%%%%%%%%%%%%%%%%%%%%%%%%%%%%%%%
For the 2-hyperbolic case, the de Moivre scale factor is
\begin{equation}
    r_{41} = i
\end{equation}
This leads to the following de Moivre scaling formulae:
\begin{align}
    B_{20}\left( i  z\right) &= \sum_{n = 0}^{\infty} \frac{(-1)^{n}}{\Gamma\left(2n + 1\right)} z^{2n} = A_{20}(z) = \cos(z) \\
    B_{21}\left( i  z\right) &= i \sum_{n = 0}^{\infty} \frac{(-1)^{n}}{\Gamma\left(2n + 2\right)} z^{2n+1} = iA_{21}(z) = i \sin(z)
\end{align}
These are the familiar relations between hyperbolic and trigonometric functions.

A second de Moivre scaling gives parity formulae:
\begin{align}
    B_{20}\left( -z\right) &= \sum_{n = 0}^{\infty} \frac{1}{\Gamma\left(2n + 1\right)} z^{2n} = B_{20}(z) \\
    B_{21}\left( -z\right) &= -\sum_{n = 0}^{\infty} \frac{1}{\Gamma\left(2n + 2\right)} z^{2n+1} = -B_{21}(z)
\end{align}
Finally, the third de Moivre scaling gives:
\begin{align}
    B_{20}\left( -i  z\right) &= \sum_{n = 0}^{\infty} \frac{(-1)^{n}}{\Gamma\left(2n + 1\right)} z^{2n} = A_{20}(z) \\
    B_{21}\left( -i  z\right) &= -i \sum_{n = 0}^{\infty} \frac{(-1)^{n}}{\Gamma\left(2n + 2\right)} z^{2n+1} = -iA_{21}(z)
\end{align}
%%%%%%%%%%%%%%%%%%%%%%%%%%%%%%%%%%%%%%%%%%%%%%%%%%%%%%%%%%%%%%%%%%%%%%%%%%%%%%%%
\subsection{Differentiation Formulae}
%%%%%%%%%%%%%%%%%%%%%%%%%%%%%%%%%%%%%%%%%%%%%%%%%%%%%%%%%%%%%%%%%%%%%%%%%%%%%%%%
The first-order differentiation formulae are
\begin{align}
    \partial B_{20} &= B_{21} & \partial B_{21} &= B_{20}
\end{align}
The second-order differentiation formulae are
\begin{align}
    \partial^{2} B_{20} &= B_{20} & \partial^{2} B_{21} &= B_{21}
\end{align}
%%%%%%%%%%%%%%%%%%%%%%%%%%%%%%%%%%%%%%%%%%%%%%%%%%%%%%%%%%%%%%%%%%%%%%%%%%%%%%%%
\subsection{Wronskian}
%%%%%%%%%%%%%%%%%%%%%%%%%%%%%%%%%%%%%%%%%%%%%%%%%%%%%%%%%%%%%%%%%%%%%%%%%%%%%%%%
With two 2-hyperbolic functions you can have only one $2 \times 2$ Wronskian:
\begin{equation}
    W_{201} = \begin{vmatrix}
        B_{20} & B_{21} \\ \partial B_{20} & \partial B_{21}
    \end{vmatrix} = \begin{vmatrix}
        B_{20} & B_{21} \\ B_{21} & B_{20}
    \end{vmatrix} = \left( B_{20} \right)^{2} - \left( B_{21} \right)^{2}
\end{equation}
Note that the Wronskian matrix is a Toeplitz matrix.
%%%%%%%%%%%%%%%%%%%%%%%%%%%%%%%%%%%%%%%%%%%%%%%%%%%%%%%%%%%%%%%%%%%%%%%%%%%%%%%%
\subsection{Algebraic Formulae}
%%%%%%%%%%%%%%%%%%%%%%%%%%%%%%%%%%%%%%%%%%%%%%%%%%%%%%%%%%%%%%%%%%%%%%%%%%%%%%%%
Recall that
\begin{equation}
    r_{40} + r_{42} = 0
\end{equation}
Let $V$ be a square matrix such that
\begin{equation}
    V^{2} = I
\end{equation}
with $I$ the identity matrix. Then
\begin{equation}
    \exp\left(r_{40} z V\right) \exp\left(r_{42} z V\right) = I
\end{equation}
This translates to
\begin{equation}
    \left( B_{20} \right)^{2} - \left( B_{21} \right)^{2} = 1
\end{equation}
which is the familiar hyperbola relation:
\begin{equation}
    \cosh^{2}(z) - \sinh^{2}(z) = 1
\end{equation}
Note that this means that $W_{201} = 1$.
%%%%%%%%%%%%%%%%%%%%%%%%%%%%%%%%%%%%%%%%%%%%%%%%%%%%%%%%%%%%%%%%%%%%%%%%%%%%%%%%
\subsection{Finite Quotient}
%%%%%%%%%%%%%%%%%%%%%%%%%%%%%%%%%%%%%%%%%%%%%%%%%%%%%%%%%%%%%%%%%%%%%%%%%%%%%%%%
Note that, for any real value $x$, you have
\begin{equation}
    B_{20}(x) \geq 1
\end{equation}
Thus, the following quotient is finite along $\mathbb{R}$:
\begin{equation}
    Q_{210}(z) \equiv \frac{B_{21}(z)}{B_{20}(z)} = \tanh(z)
\end{equation}
The first derivative of this quotient gives
\begin{equation}
    \partial Q_{210} = 1 - \left( Q_{210} \right)^{2}
\end{equation}
The second derivative of this quotient gives
\begin{equation}
    \partial^{2} Q_{210} = -2 Q_{210} + 2\left( Q_{210} \right)^{3}
\end{equation}
%%%%%%%%%%%%%%%%%%%%%%%%%%%%%%%%%%%%%%%%%%%%%%%%%%%%%%%%%%%%%%%%%%%%%%%%%%%%%%%%
\subsection{Reciprocals}
%%%%%%%%%%%%%%%%%%%%%%%%%%%%%%%%%%%%%%%%%%%%%%%%%%%%%%%%%%%%%%%%%%%%%%%%%%%%%%%%
The reciprocals are
\begin{align}
    R_{20}(z) \equiv \frac{1}{B_{20}(z)} &= \operatorname{sech}(z) \\
    R_{21}(z) \equiv \frac{1}{B_{21}(z)} &= \operatorname{csch}(z)
\end{align}
Since $B_{20}$ is non-zero along $\mathbb{R}$, $R_{20}$ is a good candidate for a probability distribution. Indeed,
\begin{equation}
    \int\limits_{-\infty}^{\infty} \mathrm{d}x R_{20}(x) = \pi
\end{equation}
The reciprocal of the above quotient is
\begin{equation}
    Q_{201}(z) \equiv \frac{1}{Q_{210}(z)} = \frac{B_{20}(z)}{B_{21}(z)} = \operatorname{coth}(z)
\end{equation}
The first derivative of these reciprocals give
\begin{align}
    \partial R_{20} &= -R_{20} Q_{210} \\
    \partial R_{21} &= -R_{21} Q_{201} \\
    \partial Q_{201} &= 1 - \left( Q_{201} \right)^{2}
\end{align}
%%%%%%%%%%%%%%%%%%%%%%%%%%%%%%%%%%%%%%%%%%%%%%%%%%%%%%%%%%%%%%%%%%%%%%%%%%%%%%%%
\subsection{Cardinal Function}
%%%%%%%%%%%%%%%%%%%%%%%%%%%%%%%%%%%%%%%%%%%%%%%%%%%%%%%%%%%%%%%%%%%%%%%%%%%%%%%%
You can define one cardinal function:
\begin{equation}
    C_{21}(z) \equiv \frac{B_{21}(z)}{z} = \sum_{n = 0}^{\infty} \frac{1}{\Gamma(2n+2)} z^{2n}
\end{equation}
This function is non-zero over $\mathbb{R}$.

The reciprocal of the cardinal function,
\begin{equation}
    D_{21}(z) \equiv \frac{1}{C_{21}(z)} = \frac{z}{B_{21}(z)}
\end{equation}
is another good candidate for a probability distribution.